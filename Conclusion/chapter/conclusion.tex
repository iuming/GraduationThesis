\section{总结}
近年来,反演法重构辐射场在辐射场重构中的研究越来越多,辐射场重构是辐射场可视化技术中的关键环节,可应用于核设施退役工程、核辐射安全科普以及矿井辐射剂量监测等等。本论文基于多层B样条插值方法和克里金插值方法提出一种适用于三维空间辐射场的插值重构方法,该方法利用有限个离散的辐射场数据能够重构出整个$ \gamma $辐射场,并提供可视化数据接口,使得重构辐射场能够以图像方式展示。主要研究内容如下:

\begin{enumerate}
    % \renewcommand{\labelenumi}{(\theenumi)}
    \item 基于多层次B样条插值和克里金插值提出一种空间辐射场三维重构方法,能够实现三维区域内散乱数据重构。该重构方法将多层B样条插值与克里金插值结果进行结合,能够在某些区域减少重构的相对偏差,从而提高了辐射场重构的精度。一般地,采用本论文提出的辐射场重构方法能够重构出较为满意的辐射场。
    \item 利用蒙特卡洛应用软件包Geant4模拟四种不同类型辐射场,通过对该四种辐射场进行重构,分别探究源项数量、辐射场空间状况以及重构测点数量对辐射场插值重构的影响。结果表明,源项数量对重构相对偏差影响较小;辐射场空间状况对重构相对偏差有一定影响;重构测点数量对辐射场插值起决定性影响。
    \item 将本论文提出的辐射场重构方法与多层B样条插值重构方法、克里金插值重构方法相比较,在单点源无屏蔽空间辐射场中,本论文提出的插值重构方法略好于另外单一插值重构方法;在辐射场剂量分布平缓的区域,多层B样条插值重构效果较好;在辐射场剂量分布变化较大的区域,克里金插值重构效果较好。在重构时间方面,克里金插值重构最快,本论文提出的插值重构其次,多层B样条插值重构最慢,但三种重构方法重构时间均在工程应用可接受范围内。
\end{enumerate}

上述研究结果表明,本论文提出的空间辐射场三维重构方法具有现实意义。由于时间、精力和知识储备的限制,还有很多不够完善的地方,在以后的工作中,可以对以下几个方面做进一步研究:

\begin{enumerate}
    \item 将本文提出的辐射场插值重构方法在实际辐射场中进行应用,分析其在实际应用中的偏差大小,进一步验证其可行性;
    \item 改进本论文辐射场插值重构中多层B样条插值和克里金插值的结合方式,将其比重与辐射场分布梯度关联,提高其插值精度;
    \item 引入更多的插值重构方法,例如有限元法、三角划分法等等,分析以上方法在哪些辐射场区域分布下精度较高,将其进行组合;
    \item 将机器学习算法应用于辐射场重构领域中,分析其可行性并与现有重构方法对比;
\end{enumerate}