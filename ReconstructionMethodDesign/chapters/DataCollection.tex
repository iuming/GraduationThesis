\section{辐射场数据获取及存储结构}
辐射场数据获取方式包括实验测量和计算机模拟。对于$ \gamma $空间辐射场测量,通常采用便携式$ \gamma $剂量率检测仪,但实验测量获取辐射场数据在考虑测量人员辐照量的情况下,实验空间内距离放射源较远的位置剂量率难以与本底区分,因此本论文用于验证空间辐射场重构方法所获取的辐射场数据,全部采用计算机模拟方法获得。

目前,计算机模拟方法中使用最多的是蒙特卡洛法(Monte Carlo method)\textsuperscript{\cite{gould1988introduction}},该方法也称为统计模拟方法,其方法是以概率统计理论为基础,获得模拟参数近似解的数值计算方法。该方法最早在核科学领域的应用是第二次世界大战时期,美国洛斯阿拉莫斯国家实验室(Los Alamos National Laboratory,简称LANL)使用蒙特卡洛方法解决核武器研发和制造过程中核燃料中子随机扩散概率问题。经过多年的发展,目前LANL已经开发出了第六代蒙特卡洛粒子输运代码(Monte Carlo N-Particle,简称MCNP),该仿真工具为辐射防护和剂量测定、辐射屏蔽、射线照相、医学物理学、核临界安全性、探测器设计和分析、核油测井、加速器目标设计、裂变和聚变反应堆的设计、净化及退役等应用领域提供了必不可少的数值模拟。同样,欧洲核子研究组织(CERN)也在20世纪90年代开发蒙特卡洛应用软件框架Geant4,其用于模拟粒子在物质中输运的物理过程,由于其开源的特点,Geant4发展至今已开发出10.7版本。Geant4在涉及微观粒子与物质相互作用的诸多领域获得了广泛应用,例如空间应用、微电子学、辐射医学、屏蔽计算。

本论文辐射场数据通过Geant4数值模拟获取,基于C++语言,利用面向对象程序设计对Geant4蒙卡模拟框架分别在几何、跟踪、探测器相应、运行管理等方面进行重新构造,模拟出本论文所需的空间辐射场各个位置剂量数据。

对于辐射场模拟数据,采用csv格式进行存储并存储为三个文件:
\begin{enumerate}
\item data.csv:用于存储测量点数据(x, y, z, value)
\item points.csv:用于存储重构点位置数据(x, y, z)
\item values.csv:用于存储重构点辐射场剂量数据(value)
\end{enumerate