\section{空间辐射场可视化}
辐射场可视化是对空间辐射场重构方法直观性的验证。相比于辐射场重构点数据,以图形、图像形式展现空间辐射场更直观。对于辐射场重构后的应用场景,例如核设施退役检修,辐射场可视化更容易对维修人员进行ALARA设计;矿井内重构辐射分布,以可视化的形式展示更加有利于矿工做出应对措施;公众进行核科普宣传时,以图形、图像甚至VR、AR形式对公众进行展示辐射场,更能提高公众对辐射防护、核安全的学习兴趣。

三维空间辐射场数据为4D数据(x,y,z,value),4D数据可视化可通过现有的虚拟现实编辑软件实现\textsuperscript{\cite{张永领2020反应堆退役三维辐射场实时计算及可视化}},或者利用一些工具框架进行实现,例如:基于matlab程序设计语言的scatter3等函数;基于python程序设计语言中matplotlib库中的相关函数;基于C++程序设计语言的OpenGL框架、ROOT框架等等。本论文采用的可视化工具为ROOT数据处理框架,ROOT是CERN开发的数据处理框架,可以用于数据存储、访问、处理以及绘制,并且提供交互式运行界面以及可创建图形化用户界面,还提供与其他程序设计语言的接口(Python、R)。