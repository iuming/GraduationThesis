\section{本章小结}
本章对辐射场插值重构方法进行了详细的验证,并且将本论文提出辐射场插值重构方法与多层B样条插值重构方法和克里金插值重构方法进行比较。本章分别探究了源项数量、辐射场空间状况以及测点数量对辐射场插值重构的影响,结果表明源项数量对辐射场插值有一定影响;空间屏蔽物对重构结果的影响比源项数量大;对辐射场插值重构起决定性影响的为初始测点数量。当辐射场测点数量低于一定数量时,插值重构效果都难以达到满意的效果。

通过将本论文提出的插值重构方法与多层B样条插值方法和克里金插值重构方法比较,发现在辐射场剂量分布较为平缓的区域,多层B样条插值重构方法重构出的辐射场效果较好;在辐射场剂量分布变化梯度较大的区域,克里金插值重构方法重构出的辐射场效果较好。总体来说,本论文提出的插值重构效果在不同区域均能重构出不错的效果。三种插值重构方法在重构时间上均能满足应用要求。