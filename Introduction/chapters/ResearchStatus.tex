\section{国内外研究现状}
相比于基于散乱数据插值方法重构辐射场,国内外当前对辐射场重构研究大部分研究都是基于正演方法。法国原子能委员会(CEA)开发基于虚拟现实技术的辐射剂量评估软件NARVEOS\textsuperscript{\cite{osti_22030167}},该软件基于点核积分法对辐射场进行重构;比利时核能研究中心(SCK·CEN)开发基于点核积分法和蒙卡抽样的辐射防护最优化工具VISIPLAN\textsuperscript{\cite{vermeersch2000software}};中国科学院核能安全技术研究所(FDS)开发了基于蒙特卡洛方法的虚拟仿真平台SuperMC/RVIS\textsuperscript{\cite{hevirtual}}。

由于正演法重构辐射场的点核积分法和蒙特卡洛法都必须在了解源项信息以及空间场结构信息后才能求解,而且点核积分法不适用于复杂的源项和空间场,蒙特卡洛方法对屏蔽较厚的场景无法得出可靠的结果以及计算时间过长,近年来,国内外开始研究基于散乱数据插值法的空间辐射场重构方法。俄罗斯科学院Krasovskii数学与力学研究所Aleksey M.Grigoryev基于径向基函数插值对辐射场进行插值重构并基于重构辐射场计算路线最优化问题\textsuperscript{\cite{Grigoryev2020Determination}};英国布里斯托尔廷德尔大道布里斯托尔大学物理学院HH Wills物理实验室Samuel R. White基于投影线性重建(PLR)算法对源项进行定位\textsuperscript{\cite{s21030807}};中国工程物理研究院赛雪基于Multiquadric散乱数据插值方法对辐射场可视化进行研究\textsuperscript{\cite{赛雪2017基于散乱数据插值方法的γ辐射场可视化技术研究}},证明了基于Multquadric方法对辐射场重构是可行的;华南理工大学电力学院王壮提出一种基于网格函数插值方法\textsuperscript{\cite{WANG201827}},通过点源辐射场数据进行重构,验证了其方法的可行性。

尽管近些年来国内外都有对基于散乱数据插值方法的辐射场重构进行研究,但是他们都是基于单一的插值方法并且仅在二维空间辐射场上对其方法验证其可行性,而实际情况下辐射场为三维空间,基于三维空间辐射场数据对插值重构方法进行验证更能说明其可行性。