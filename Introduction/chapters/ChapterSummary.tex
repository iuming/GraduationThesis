\section{本论文章节安排}
本文通过综合分析空间插值重构方法在各个领域的应用以及国内外研究现状,在传统单一插值重构方法上进行改进,提出结合优化样条插值重构法和克里金插值重构法的新型插值重构方法,通过对两种重构方法的结合,获得更符合实际辐射场的插值重构方法。结果表明,该方法重构出的辐射场相比于传统方法重构出的辐射场更接近实际辐射场。同时,在采用新型插值重构方法时,选取离散采样点时更有趋向性。

第一章主要介绍本论文的研究背景和研究意义以及空间辐射场重构的插值方法,分析国内外空间辐射场重构研究现状,提出本论文的主要研究内容和章节安排。

第二章主要介绍样条插值重构方法,分别对B样条插值和多层B样条插值的原理进行阐述,给出BA算法和MBA算法的具体计算流程。

第三章主要介绍克里金插值重构方法,具体阐述了克里金插值方法的具体理论,基于普通克里金模型推导出克里金插值方法的矩阵形式。

第四章提出基于样条插值方法和克里金插值方法的新型插值重构方法,通过程序框图具体说明新型辐射场重构方法的过程。

第五章基于蒙特卡洛应用工具框架Geant4,模拟几组不同应用场景下的空间辐射场数据,用于验证空间辐射场重构方法的合理可行性,并对比其与传统重构方法的重构效果。

最后总结全文,对空间辐射场重构方法的进一步研究提出展望和建设性意见。