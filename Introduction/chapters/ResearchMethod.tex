\section{研究方法和内容}
空间插值重构方法有多种分类方式\textsuperscript{\cite{李海涛2019空间插值分析算法综述}}:按照插值区域范围分,有整体插值、局部插值、边界内插法等。其中整体插值是利用研究区域的所有散乱数据点进行全局特征拟合,采用整体插值方法时,整个区域的数值会影响单个插值点的数值,同理单个散乱数据点的数值增加、减少或删除对整个区域的特征拟合都会造成影响,代表性的整体插值方法有趋势面分析插值方法等;局部插值是利用临近数据点来预测插值点的值:首先定义邻域或搜索范围,然后在该区域内搜索散乱数据点,再对该区域内数据点选择插值函数进行拟合,最后通过计算插值函数得到预测点的值,代表性的局部插值方法有样条插值法、反距离权重插值法和克里金插值法等;边界内插法假设任何数值变化都发生在区域边界上,并且边界的变化是均匀的、同质的,代表性的边界内插方法有泰森多边形法等。

按照插值的标准分,可以分为确定性插值、地统计插值:确定性插值法主要采用数学工具,利用计算插值函数的方法来进行插值,这种方式用来研究某区域内部的相似性,其代表插值法有样条插值法、反距离加权插值法等;地统计插值是基于空间自相关性的,由观测数据产生具有统计关系的曲面,代表插值法有克里金插值法等。按照插值的精度分,可以分为精确插值、近似插值。精确插值重构出包括所有散乱数据点的辐射场;近似插值重构出不包含所有散乱数据点的辐射场。

本论文采用的空间插值重构方法是基于样条插值法和克里金插值法,对于给定的辐射场离散数据,首先分别通过样条插值方法和克里金插值方法重构出相应的辐射场数据;然后对比两种方法重构出的辐射场数据,选取重构结果数据相差较大的区域进行重新采样测量,直到两种方法重构出的辐射场结果大致相同;最后将两种方法重构的辐射场数据进行组合,得到重构空间辐射场分布。