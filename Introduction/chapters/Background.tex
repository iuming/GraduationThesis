\section{研究背景与意义}
空间辐射场三维重构方法是辐射场可视化仿真技术的一项重要技术。辐射场可视化仿真技术不仅能够应用在核设施退役工程,将辐射场的分布以可视化的形式显示在虚拟场景中,评估施工人员所接受的辐射剂量\textsuperscript{\cite{wang2020reconstruction}};还能在核与辐射安全科普工作中发挥积极的作用,将辐射场的剂量值、分布等定量信息展现给公众,易于科普工作者与公众进行互动沟通,提高公众对核科学的认识\textsuperscript{\cite{曹亚丽2014科学传播模式在我国核与辐射安全科普工作中的应用}}。随着计算机硬件及相关技术的发展,可视化和虚拟现实技术在辐射防护领域得到了迅速发展\textsuperscript{\cite{chen2021visualization}},进而辐射场重构技术的方法研究就显得格外迫切。

总所周知,核辐射是一种弱致癌物,例如氡(Rn)作为天然气态放射性核素,氡及其衰变子体已成为肺癌的第二大诱因\textsuperscript{\cite{Böhm2020Radon}}。根据联合国原子辐射影响科学委员会(UNSCEAR)对氡及其子体的流行病学和计量学审查研究,矿工的肺癌终生风险率每工作一个月提高$ 2.4 \sim 7.5 \times 10^{-4} $(用WLM表示,Working Level Month)\textsuperscript{\cite{John2021Lung}}。2015年德国萨克森辐射防护局检察员Jörg Dehnert对矿井内252名矿工进行个人剂量监测,其中最高有效年剂量高达$ 14.4 mSv $,并且改进矿井内辐射防护措施:通过对矿井内放射性气体浓度进行监测,根据浓度调节矿井内风机转速\textsuperscript{\cite{dehnert2020radon}}。放射性浓度高的地区除了矿山矿井,还有地下建筑、窑洞、工业废渣建筑室内以及随着核工业技术的发展,核设施周围都存在不同强度的辐射场。这些辐射场都存在着一些特点:放射性物质分布不明、源项分布或场景结构复杂。

对于复杂的辐射场,目前常用的重构方法有正演方法和反演方法。正演方法是通过对辐射传输方程进行求解,在了解放射源基本信息的基础上,构造准确的系统模型进行粒子运输模拟,从而将空间辐射场进行重构;反演方法是在未知放射源基本信息的情况下,通过实际测量获得有限、离散的采样数据进行分析和空间重构,从而获得完整的辐射场分布数据。正演方法目前常用的一些算法有蒙特卡洛法\textsuperscript{\cite{MAJER2019108824}}和点核积分法\textsuperscript{\cite{ZHANG2021108179}}等,其中蒙特卡洛法是通过随机性方法对辐射传输方程进行求解计算,点核积分法是通过确定论方法对辐射传输方程进行求解。反演方法也指散乱数据重构方法,目前主要有插值\textsuperscript{\cite{WANG201827}}和逼近,常用的插值算法包括多项式插值、径向基插值、反距离权重插值等,常用的逼近算法包括最小二乘法、最小立方法等。

本文研究的重构方法基于反演方法,在未知源项信息的空间辐射场中,基于测量的有限个离散数据点,利用插值重构方法快速重构出空间辐射场的剂量分布情况。本文研究方法能够应用于监测工作场所辐射场分布、预估未知空间场源项位置、辅助核设施现场辐射防护计算等,拥有广阔的使用场景和应用领域。