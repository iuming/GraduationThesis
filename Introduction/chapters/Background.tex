\section{研究背景与意义}
空间辐射场三维重构方法是辐射场可视化仿真技术的一项重要技术。辐射场可视化仿真技术不仅能够应用在核设施退役工程,将辐射场的分布以可视化的形式显示在虚拟场景中,评估施工人员所接受的辐射剂量\textsuperscript{\cite{wang2020reconstruction}};还能在核与辐射安全科普工作中发挥积极的作用,将辐射场的剂量值、分布等定量信息展现给公众,易于科普工作者与公众进行互动沟通,提高公众对核科学的认识\textsuperscript{\cite{曹亚丽2014科学传播模式在我国核与辐射安全科普工作中的应用}}。随着计算机硬件及相关技术的发展,可视化和虚拟现实技术在辐射防护领域得到了迅速发展\textsuperscript{\cite{chen2021visualization}},进而辐射场重构技术的方法研究就显得格外迫切。

总所周知,核辐射是一种弱致癌物,例如氡(Rn)作为天然气态放射性核素,氡及其衰变子体已成为肺癌的第二大诱因\textsuperscript{\cite{Böhm2020Radon}}。根据联合国原子辐射影响科学委员会(UNSCEAR)对氡及其子体的流行病学和计量学审查研究,矿工的肺癌终生风险率每工作一个月提高$ 2.4 \sim 7.5 \times 10^{-4} $(用WLM表示,Working Level Month)\textsuperscript{\cite{John2021Lung}}。2015年德国萨克森辐射防护局检察员Jörg Dehnert对矿井内252名矿工进行个人剂量监测,其中最高有效年剂量高达$ 14.4 mSv $,并且改进矿井内辐射防护措施:通过对矿井内放射性气体浓度进行监测,根据浓度调节矿井内风机转速\textsuperscript{\cite{dehnert2020radon}}。

放射性浓度高的地区除了矿山矿井,还有地下建筑、窑洞、工业废渣建筑室内以及随着核工业技术的发展,核设施周围都存在不同强度的辐射场。这些辐射场都存在着一些特点:放射性物质分布不明、源项分布或场景结构复杂。而传统的蒙特卡洛方法和点核积分方法计算辐射场分布都需要利用源项数据以及空间内场景信息,不能对以上场景的辐射场进行重构。