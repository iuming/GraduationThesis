\begin{abstract}
    辐射场可视化仿真技术是辐射防护中的一项重要技术,其应用场景十分广泛,例如在核设施退役工程中评估工作人员照射的辐射剂量、在核辐射安全科普工作中向公众展示辐射分布信息、在矿井窑洞等辐射剂量高的地域监测放射性浓度等等。空间辐射场重构方法是辐射场可视化仿真技术中最核心的技术,因此找寻一种重构速度快、精度高、适用性好的辐射场重构方法就显得尤为重要。

    针对上述问题,本文在调研国内外辐射场插值重构研究现状的基础上,针对插值重构方法单一的问题进行深入研究,提出一种基于多层B样条插值方法和克里金插值方法的空间辐射场插值重构方法,主要研究内容和创新点如下:

    \begin{enumerate}
        \item 提出并实现了基于多层B样条插值方法和克里金插值方法的辐射场重构方法,详细地分析了多层B样条插值原理和克里金插值原理,通过将两种插值方法结合,提出一种精度更高、适用性更好的辐射场重构方法;
        \item 利用Geant4设计并构建了四种不同类型的辐射场模型,用于验证辐射场重构方法的可行性,分别探究了源项数量、空间状况以及测点数量对辐射场重构方法的影响;
        \item 将本论文提出的辐射场重构方法与多层B样条插值重构方法、克里金插值重构方法进行辐射场重构效果对比,进一步说明该辐射场重构方法的可行性。
    \end{enumerate}

    \textbf{关键词:} 辐射场重构;多层B样条插值;克里金插值;辐射场可视化
\end{abstract}

\newpage
\ctexset{
    abstractname = {Abstract},
}
\begin{abstract}
    Radiation field visualization simulation technology is an important technology in radiation protection, which has a variety of real world applications, such as evaluating the radiation dose of workers in nuclear facility decommissioning projects, displaying radiation distribution information to the public in nuclear radiation safety science work, Monitoring the concentration of radioactivity in areas with high radiation doses such as mines and cave dwellings, etc. The space radiation field reconstruction method is the core technology of the radiation field visualization simulation technology. Therefore, it is particularly important to find a radiation field reconstruction method with fast reconstruction speed, high accuracy and good applicability.

    In response to the above problems, this article investigates the current status of radiation field interpolation reconstruction at home and abroad, and conducts in-depth research on the single problem of interpolation reconstruction method, and proposes a multi-layer B-spline interpolation method and Kriging interpolation method. The main research contents and innovations of the spatial radiation field interpolation reconstruction method are as follows:

    \begin{enumerate}
        \item The radiation field reconstruction method based on the Multilevel B-spline interpolation method and the Kriging interpolation method is proposed and implemented. The principle of the Multilevel B-spline interpolation and the Kriging interpolation principle are analyzed in detail. By combining the two interpolation methods, To propose a radiation field reconstruction method with higher accuracy and better applicability;
        \item Four different types of radiation field models were designed and constructed using Geant4 to verify the feasibility of the radiation field reconstruction method. The influence of the number of source items, space conditions, and the number of measuring points on the radiation field reconstruction method were explored respectively;
        \item The radiation field reconstruction method proposed in this paper is compared with the Multilevel B-spline interpolation reconstruction method and the Kriging interpolation reconstruction method to compare the radiation field reconstruction effect, and further illustrate the feasibility of the radiation field reconstruction method.
    \end{enumerate}
    \textbf{Keywords:} Radiation field reconstruction; Multilevel B-spline interpolation; Kriging interpolation; Radiation field visualization
\end{abstract}
